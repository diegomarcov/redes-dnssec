\chapter{Conclusiones}

En un principio, todo lo relacionado con computadoras y las redes de
comuniacaciones estaba destinado a ambientes acad\'emicos y operaban bajo la
idea de compartir informaci\'on \'unicamente. Hoy en d\'ia el panorama ha
cambiado considerablemente. Internet sigue teniendo como principal objetivo el
compartir informaci\'on, pero tiene muchos otros objetivos que corren a la par
de este. Entre ellos podemos encontrar a la promoci\'on a trav\'es de la
publicidad, u otros objetivos claramente monetarios.

La monetarizaci\'on de internet ha cambiado y deformado los principios con los
que surgi\'o, y esto trajo consigo una serie de problemas para los usuarios.

Virus o ``malware'' que buscan propagarse por la red e inyectar anuncios o robar
informaci\'on, son uno de los ejemplos m\'as conocidos de esto, y
lamentablemente el hecho de que $la red$ haya sido pensada en un \'ambito
acad\'emico bajo objetivos nobles hizo que el diseño de los protocolos sobre
los cuales se montan los sitemas no tengan en cuenta la posibilidad de que
alguien haga mal uso de los recursos.

A lo largo de los \'ultimos 10 a\~nos ha habido una tendencia de agregar una
capa protegida criptogr\'aficamente a los $stacks$ de protocolos, en su
mayor\'ia utilizando alguna forma del ahora conocido SSL. HTTPS est\'a
comenzando a ser el standard en sitios web con informaci\'on ``sensible'' para
el usuario. POP3, SMTP, IMAP, FTP, todos protocolos que transfer\'ian todos sus
datos en texto plano, se comenzaron a utilizar de la mano con SSL o SSH. El
objetivo de esta nueva capa ``protegida'' es evitar el $sniffing$ de datos en
la red por una tercera parte.

Entre los distintos protocolos que surgieron años atr\'as, se encuentra DNS.
Este protocolo, si bien no transporta informaci\'on que a primera vista se
considere sensible, igualmente est\'a necesitando protecciones
criptogr\'aficas ya que de este servicio depende las direcciones hacia las
cuales los usuarios de la red nos dirigimos cuando tratamos de utilizar alg\'un
servicio direccionado con una URL.

----------------

De este hecho se desataron dos problematicas. Por un lado, DNS es un protocolo
muy complejo y descentralizado de los m\'as usados en la actualidad de forma
``indirecta'' (no es accedido expl\'icitamente). Esto hace que una
modificaci\'on en el mismo sea de alta complejidad.

Y por el otro lado, el hecho de que la tendencia sea usar un protocolo del
estilo de SSL bas\'andose en el modelo de confianza transitiva basado en firmas
digitales, hace que los simientos de lo que se llama DNSSEC sean
considerablemente d\'ebiles. 

El modelo de confianza transitiva en las firmas digitales tiene muchos
problemas, como se lo ve en la actualidad con HTTPS y las Entidades
Certificantes. La confianza en una firma se reduce a un checkeo recursivo de
quien firma que la firma actual es de quien dice ser, hasta llegar a una firma
``root'', cuya confianza est\'a dada por una arbitrariedad establecida por
concenso general, es decir, si la cadena de firmas termina en la firma de una
entidad en la que se conf\'ia porque se sabe que la misma corrobora
rigurosamente a quien certifica no significa que en la actualidad capitalista
no se puedan saltear dichas ``barreras'' y que la instituci\'on en cuesti\'on
certifique una entidad bajo datos falsos.

De cualquier manera, este modelo presenta problemas aun mayores. Aun cuando la
entidad certificante sea de confianza, un peque\~no error en la infraestructura
de la entidad puede desatar un fallo de seguridad que puede llevar a un
compromiso de las herramientas utilizadas con el objetivo de emitir firmas
fraudulentas sin el consentimiento de la entidad, y usuarios caigan v\'ictimas
de esto sin que puedan darse cuenta.

Adem\'as de este hecho, como se vio, DNSSEC por dise\~no no est\'a pensado para
proteger de muchos ataques conocidos. E indirectamente acent\'ua otros que
pueden llevar a graves consecuencias.

Peor aun, asumiendo que DNSSEC protege los \'angulos que buscamos proteger, el
``deployment'' del mismo resulta tan complejo que luego de m\'as de 5 años desde
que se comenz\'o a especificar formalmente el protocolo.

En conclusi\'on, los objetivos perseguidos por DNSSEC son copados, pero es una
bosta.