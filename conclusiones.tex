\chapter{Conclusiones}
Podemos decir que en DNS se desataron dos problem\'aticas importantes: por un lado, al ser un protocolo muy utilizado, complejo y descentralizado que no se accede expl\'icitamente, realizar modificaciones en \'el es un problema de alta complejidad. Por el otro, el hecho de que la tendencia sea usar un protocolo del estilo de SSL a trav\'es del modelo de confianza transitiva basado en firmas digitales, hace que los cimientos de DNSSEC sean considerablemente d\'ebiles.

El modelo de confianza transitiva en las firmas digitales tiene muchos problemas, como puede verse en la actualidad con HTTPS y las Entidades Certificantes. La confianza en una firma se reduce a un \textit{checkeo} recursivo de que quien certifica la firma actual es un certificado de confianza, hasta llegar a una firma ``root'', cuya confianza est\'a dada por una arbitrariedad establecida por concenso general; esto significa que existe la posibilidad de que una cadena de firmas termine en la firma de una entidad en la que se conf\'ia porque se cree que \'esta corrobora rigurosamente a qui\'en certifica, pero en verdad la instituci\'on en cuesti\'on certifique una entidad bajo datos falsos.

De cualquier manera, el modelo presenta problemas a\'un mayores. A\'un cuando la entidad certificante sea de confianza, un peque\~no error en la infraestructura de la entidad puede desatar un fallo de seguridad que puede resultar en un compromiso de las herramientas utilizadas con el objetivo de emitir firmas fraudulentas sin el consentimiento de la entidad, y que los usuarios resulten v\'ictimas de esto sin que puedan darse cuenta.

Adem\'as de este hecho, como se vi\'o anteriormente, DNSSEC por dise\~no no est\'a pensado para proteger de muchos ataques ya conocidos, e indirectamente acent\'ua otros que pueden llevar a graves consecuencias. \'Esto, sumado a la falta de confidencialidad inherente al protocolo, hacen que las ventajas obtenidas en el marco te\'orico est\'en lejos de mostrar sus frutos en la pr\'actica.

Por \'ultimo, a\'un en el caso de que DNSSEC protegiera todos los \'angulos que se busca proteger, el ``deployment'' de \'este resulta tan complejo que luego de m\'as de 5 a\~nos desde que se comenz\'o a especificar formalmente el protocolo, a\'un son unos pocos servidores los que lo utilizan, y usualmente s\'olo de manera opcional. Si consideramos adem\'as los bugs encontrados en muchas de las implementaciones consideradas ``seguras'', que resultaron en que muchos datos te\'oricamente protegidos pudieran accederse con facilidad mediante ataques sin mayor complejidad, podemos concluir que a pesar de las buenas intenciones del planteo de DNSSEC, el protocolo contin\'ua sin ser una buena fuente de seguridad para los usuarios.