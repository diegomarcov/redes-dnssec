\chapter{DNSSEC en profunidad}
El principal concepto que trae DNSSEC de la mano es el de $Firmado de zonas$. Para estrablecer una zona firmada de una forma críptográficamente segura, se deben contar con los siguientes campos:

\begin{itemize}
  \item DNSKEY: DNS Public Key
  \item RRSIG: Resource Record Signature
  \item NSEC: Next Secure
  \item DS: Delegation Signer
\end{itemize}

Para poder utilizar esta estructura nueva de firmado de datos, cada administrador de zona debe generar un par de claves pública y privada. Esta clave privada será la que se usará para firmar conjuntos de Resourse Records dentro de la zona. Para que sea posible la verificación de la firma, el administrador puede proveer el registro DNSKEY con la clave pública que va de la mano de la privada con la que firmó los RRs.

Para cada conjunto de Registro de Recurso tiene que existir una firma en el formato de RRSIG, cuyos datos conformen los declarados en la definición del conjunto. De forma opcional, se pueden firmar más de una vez un conjunto, aun que esto no agrega gran seguridad por sobre el modelo básico.

Dado que DNS es una estructura jerárquica, el manejo de zonas y delegación de las mismas es escencial, y a la vez un gran punto de problemas para el protocolo. Para tratar de mejorar la seguridad de procedimientos como el intercambio de zonas o la delegación, se introducen RRs como NSEC y DS.

\subsection{El lado del cliente}

Desde el lado del cliente, los registros de los cuales se hablaba anteriormente son la base para la autenticación de las respuestas recibidas de un servidor DNS.

Como muchos sistemas de seguridad basada en firmado criptográfico asimétrico, DNSSEC establece la autenticación de un servidor dependiendo de la cadena de confianza de la firma que este provea. De esta forma, el cliente cuenta con un conjunto de firmas de confianza, y en base a ellos determina la confianza de las respuestas obtenidas. Este conjunto de certificados $base$ representa lo que se suele denominar $Trust Anchor$, y cada firma que posea una cadena de confianza que termine o que en algún punto referencie algún elemento de nuestro $Trust Anchor$ será de confianza.

Para establecer la relación en la cadena de confianza, se utilizan los registros descriptos con anterioridad: NSEC y DS. Si no se cuenta con un $Trust Anchor$ con al menos una clave pública disponible, no será posible verificar ningún tipo de respuesta con la seguridad que se pretende proveer. En un caso extremo como este, es posible detectar algún tipo de manipulación de las respuestas por un atacante que no haya interceptado la consulta de la DNSKEY y la haya reemplazado por una cuya clave privada él controla, pero un atacante sofisticado podría manipular los datos de forma tal que no sea posible detectarlo.

\subsection{Consideraciones técnicas}

Los problemas de seguridad informática se pueden representar como una cadena que se busca romper. El atacante siempre tendrá como objetivo principal el eslabón más debil, por ello es que hay que cuidar cada parte al máximo.

Una de las primeras cuestiones a tener en cuenta son los algoritmos criptográficos utilizados. Si bien el standard elegido es RSA/SHA-1, se recomienda utilizar RSA/SHA-256. En cuanto a la longitud de las claves RSA, lo ideal en este momento es utilizar no menos de 4096 bits, y tiempos de efectividad no demasiado largos, ya que aquellos pares de claves con mayor tiempo de validés serán los principales objetivos de un atacante. Los sistemas criptográficos imposibles de romper son aquellos cuyas claves son utilizadas para encriptar un único conjunto de datos ($One Time Pads$), y si bien no es un extremo práctico para algo como DNSSEC, es importante encontrar un buen punto de compromiso entre seguridad y practicidad. En general, las claves RSA suelen usarse con validés de 1 año.

Por otro lado, aun cuando las llaves sean de un tamaño mayor o igual a 4096 bits, y la validés sea corta, el resguardo del la parte privada de la misma es muy importante. En general se recomienda mantenerlas en algún lugar que no sea accesible desde internet.
