\chapter{�De qu\'e nos protege DNSSEC?}
DNSSEC provee mecanismos para autentificar el origen y corroborar la integridad de los datos recibidos. Dado que el sistema propone la firma digital de todos los paquetes, resulta necesario agregar algunos nuevos registros y bits al encabezado del paquete, lo que hace que el cambio no sea trivial.\\
Es importante destacar que la firma digital corresponde a una zona de dominio, y \textit{no} a los servidores. El mecanismo tambi\'en permite la inclusi\'on de m\'ultiples claves para firmar una zona (que puede ser utilizada, por ejemplo, en caso de desear usar difrentes algoritmos para la generaci\'on de claves).\\
Un cliente puede aprender la clave p\'ublica correspondiente a una zona de dos maneras: configurando una referencia de confianza (\textit{trust anchor}), o a partir de la resoluci\'on normal utilizando DNS (utilizando un nuevo RR\footnote{Registro de Recursos} agregado, llamado \texttt{DNSKEY}) y formando una cadena de autentificaci\'on. Sin embargo, esto implica que el cliente debe tener, al menos, una referencia de confianza configurada, lo que puede resultar en otros problemas que revisaremos en el cap\'itulo siguiente.\\
Usualmente, la cadena de autentificaci\'on es construida por el cliente desde la ra\'iz de la jerarqu\'ia DNS hacia cada una de las zonas. Sin embargo, la pol\'itica de seguridad local es independiente de DNSSEC, por lo que existen diversos problemas que hacen que el funcionamiento del mecanismo no siempre sea el esperado: un cliente puede tener configuradas de manera local claves p\'ublicas (o sus respectivos hash) de confianza en lugar de utilizar la clave p\'ublica de la ra\'iz, o no proveer una configuraci\'on adecuada para dicha clave, o por alg\'un motivo arbitrario bloquear claves p\'ublicas particulares a\'un cuando \'estas est\'en correctamente verificadas.
M\'as all\'a de estas problem\'aticas, resulta evidente que DNSSEC presenta soluciones para los ataques mostrados en el segundo cap\'itulo: la intercepci\'on de paquetes puede prevenirse utilizando correctamente los checkeos de integridad de datos que DNSSEC propone. De la misma manera, los ataques del tipo \textit{ID-guessing} pueden ser detectados por cualquier cliente que compruebe las firmas digitales de los paquetes.\\
Nos enfocaremos ahora en la familia de ataques de tipo \textit{name chaining}, que fueron una de las principales motivaciones para la creaci\'on de DNSSEC. Naturalmente, la comprobaci\'on de las firmas digitales puede permitir al cliente asegurarse de que los datos asociados a un nombre realmente hayan sido insertados por la autoridad correspondiente a ese espacio de nombres. M\'as precisamente, el cliente puede determinar si la entidad que inyect\'o los datos tuvo acceso a una clave privada cuya clave p\'ublica aparece en en una locaci\'on esperada del espacio de nombres, con una cadena de firmas digitales parentales que comienza con una clave p\'ublica de la que el cliente ten\'ia conocimiento previo.\\
Por \'ultimo, si bien las firmas digitales propuestas en DNSSEC no cubren a los \textit{glue records}\footnote{Entradas creadas en la entidad que gestiona un TLD para que un dominio pueda actuar como servidor DNS}, lo que en principio permitir\'ia un ataque \textit{name chaining} utiliz\'andolos, s\'i es posible detectarlos aceptando temporalmente el \textit{glue} para obtener la versi\'on firmada de los datos, y a partir de all\'i realizar la comprobaci\'on.