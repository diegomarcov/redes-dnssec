\chapter{Introducci�n}

El objetivo de este trabajo es realizar un estudio sobre DNSSEC\footnote{Domain Name System Security Extensions}, incluyendo tanto las especifaciones que lo componen como las problem�ticas existentes a la hora de realizar su implementaci�n y puesta en marcha.

\section{DNS hoy}
DNS es un sistema de nomenclatura jer�rquica que permite asociar nombres a identificadores binarios asociados a cualquier equipo conectado a una red. Existen las llamadas ``zonas de autoridad'', que son porciones del espacio de nombres, y cada una de ellas est� compuesta por al menos un dominio. A su vez, los subdominios de cada uno de estos dominios pueden ser delegados a otras zonas de autoridad diferentes.

\subsection{Vulnerabilidades}
Tanto algunas deficiencias inherentes al protocolo DNS como algunos defectos en las implementaciones m�s comunes hacen que el sistema de DNS actual sea vulnerable a diferentes tipos de ataques. En el cap�tulo siguiente veremos algunas de estas debilidades en detalle, con especial �nfasis en \textit{cache poisoning}: una t�cnica que permite manipular informaci�n del cach� de un servidor DNS, y una de las principales motivaciones para la creaci�n de DNSSEC\cite{cert08}.

\section{�Qu� es DNSSEC?}
DNSSEC es un conjunto de especificaciones dise�adas para proveer mayor seguridad a los clientes DNS respecto a los or�genes e integridad de los datos\cite{rfc4033}. Dada la utilizaci�n global de DNS, la necesidad de dicha seguridad para hacer frente a las vulnerabilidades del sistema actual es cada vez m�s importante. La principal proposici�n de DNSSEC es firmar digitalmente todas las respuestas enviadas desde los servidores, de manera que los clientes puedan corroborar que la informaci�n recibida es correcta.
