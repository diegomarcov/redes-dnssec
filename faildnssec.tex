\chapter{¿Qu\'e problemas presenta DNSSEC?}
Para la correcta implementaci\'on y puesta en marcha de DNSSEC de manera masiva existen algunos conflictos que act\'uan a modo de traba.\\
Por un lado, es necesario introducir algunos cambios operacionales tanto para los clientes como para los servidores. Seg\'un la lista de comprobaci\'on presentada por Ekl\"{o}v y Lagerholm en un art\'iculo del IP Journal de Cisco\cite{imp2011}, podemos destacar:
\begin{itemize}
 \item Open your firewall for EDNS0 signaling and allow large DNS packets using UDP and TCP over port 53.
 \item Check the DNSSEC capabilities of all your masters and slave servers.
 \item Check the DNSSEC capabilities of your registrar and understand their requirements for the public key you are uploading.
 \item Make sure your IPAM system can handle secure delegations.
 \item Plan how to handle load balancers. Develop an automation strategy if you have a lot of zones.
 \item Plan how you will transfer your keys to a new master server if a disaster occurs.
 \item Implement a policy for DNSSEC timer settings.
\end{itemize}

Sin embargo, todos los \'items mostrados pueden ser solucionados comprendiendo las implicaciones de la implementaci\'on del sistema y planeando por adelantado. Los problemas m\'as graves aparecen a partir de las fallas de seguridad que DNSSEC no soluciona, y en algunos casos, \textbf{llega a empeorar gravemente}. Estudiaremos a continuaci\'on los considerados maś remarcables:

ACÁ VAN LOS PROBLEMAS QUE ESTÁN EN LAS SLIDES DE SEGURIDAD DEL GURÚ.