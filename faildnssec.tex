\chapter{¿Qu\'e problemas presenta DNSSEC?}
Para la correcta implementaci\'on y puesta en marcha de DNSSEC de manera masiva existen algunos conflictos que act\'uan a modo de traba.\\
Por un lado, es necesario introducir algunos cambios operacionales tanto para los clientes como para los servidores. Seg\'un la lista de comprobaci\'on presentada por Ekl\"{o}v y Lagerholm en un art\'iculo del IP Journal de Cisco\cite{imp2011}, podemos destacar:
\begin{itemize}
 \item Reconfiguraci\'on de los firewall: las \textit{queries} DNS primero intentan utilizar UDP, y posteriormente TCP si no reciben ninguna respuesta para el paquete UDP inicial \textit{o bien} la respuesta es trunca; dado que el tama\~no de los paquetes DNSSEC es mayor, la probabilidad de que algo bloquee parte de la respuesta utilizando DNSSEC es mucho m\'as grande. Para que el sistema funcione correctamente, resulta obligatorio abrir los firewall tanto para TCP como para UDP en el puerto 53. Por otro lado, como DNSSEC utiliza el standard EDNS0\footnote{Uno de los mecanismos de extensi\'on de DNS que permite, entre otras cosas, que un cliente indique que puede recibir respuestas DNS utilizando UDP con una longitud mayor a 512 bytes}, es necesario que el firewall est\'e configurado para interpretar dicho standard correctamente. De otra manera, los firewall tienden a bloquear los paquetes DNSSEC y no permiten la comunicaci\'on. Existe un estudio realizado por ICANN sobre el soporte de EDNS0 en los firewall m\'as com\'unes\cite{icann07}.
 \item Muchos TLD requieren que el registrante (ej: \textit{redes.org}) no se comunique directamente con el registro (\textit{.org}), sino que una tercer entidad (en ingl\'es llamada \textit{registrar}) es la encargada de manejar toda la comunicaci\'on relacionada a DNS y DNSSEC. Esto implica que es necesario comprobar que el \textit{registrar} en cuesti\'on tenga efectivamente soporte para DNSSEC para el TLD en cuesti\'on.
 \item Es muy usual que grandes organizaciones que requieren alta disponibilidad para sus servidores utilicen balanceadores de carga que respondan seg\'un el estado de los servicios. Sin embargo, actualmente son pocos los balanceadores que soportan DNSSEC.
\end{itemize}

Sin embargo, los \'items mostrados pueden ser solucionados comprendiendo las implicaciones de la implementaci\'on del sistema y a partir de una cuidadosa planificaci\'on. Los problemas m\'as graves aparecen a partir de las fallas de seguridad que DNSSEC no soluciona, y en algunos casos, \textbf{llega a empeorar gravemente}. Estudiaremos a continuaci\'on los considerados m\'as remarcables: (ver las transparencias del Gur\'u!!!)

\begin{itemize}
	\item DNSSEC no mejora availability
	\item DNSSEC empeora RADICALMENTE los ataques DDoS
	\item DNSSEC tiene fallas que hacen que la privacidad no mejore
	\item A\'un utilizando DNSSEC+NSEC3, hay leak de datos privados.
\end{itemize}