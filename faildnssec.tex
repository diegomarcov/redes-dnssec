\chapter{�Qu\'e problemas presenta DNSSEC?}
Para la correcta implementaci\'on y puesta en marcha de DNSSEC de manera masiva existen algunos conflictos que act\'uan a modo de traba.\\
Por un lado, es necesario introducir algunos cambios operacionales tanto para los clientes como para los servidores. Seg\'un la lista de comprobaci\'on presentada por Ekl\"{o}v y Lagerholm en un art\'iculo del IP Journal de Cisco\cite{imp2011}, podemos destacar:
\begin{itemize}
 \item Reconfiguraci�n de los firewall: las \textit{queries} DNS primero intentan utilizar UDP, y posteriormente TCP si no reciben ninguna respuesta para el paquete UDP inicial \textit{o bien} la respuesta es trunca; dado que el tama�o de los paquetes DNSSEC es mayor, la probabilidad de que algo bloquee parte de la respuesta utilizando DNSSEC es mucho m�s grande. Para que el sistema funcione correctamente, resulta obligatorio abrir los firewall tanto para TCP como para UDP en el puerto 53. Por otro lado, como DNSSEC utiliza el standard EDNS0\footnote{Uno de los mecanismos de extensi�n de DNS que permite, entre otras cosas, que un cliente indique que puede recibir respuestas DNS utilizando UDP con una longitud mayor a 512 bytes}, es necesario que el firewall est� configurado para interpretar dicho standard correctamente. De otra manera, los firewall tienden a bloquear los paquetes DNSSEC y no permiten la comunicaci�n. Existe un estudio realizado por ICANN sobre el soporte de EDNS0 en los firewall m�s com�nes\cite{icann07}.
 \item Muchos TLD\footnote{Top-Level Domain} requieren que el registrante (ej: \textit{redes.org}) no se comunique directamente con el registro (\textit{.org}), sino que una tercer entidad (en ingl�s llamada \textit{registrar}) es la encargada de manejar toda la comunicaci�n relacionada a DNS y DNSSEC. Esto implica que es necesario comprobar que el \textit{registrar} en cuesti�n tenga efectivamente soporte para DNSSEC para el TLD en cuesti�n.
 \item Es muy usual que grandes organizaciones que requieren alta disponibilidad para sus servidores utilicen balanceadores de carga que respondan seg�n el estado de los servicios. Sin embargo, actualmente son pocos los balanceadores que soportan DNSSEC.
\end{itemize}

Sin embargo, los \'items mostrados pueden ser solucionados comprendiendo las implicaciones de la implementaci\'on del sistema y a partir de una cuidadosa planificaci�n. Los problemas m\'as graves aparecen a partir de las fallas de seguridad que DNSSEC no soluciona, y en algunos casos, \textbf{llega a empeorar gravemente}. Estudiaremos a continuaci\'on los considerados m�s remarcables: (ver las transparencias del Gur�!!!)

\begin{itemize}
	\item DNSSEC no mejora availability
	\item DNSSEC empeora RADICALMENTE los ataques DDoS
	\item DNSSEC tiene fallas que hacen que la privacidad no mejore
	\item A�n utilizando DNSSEC+NSEC3, hay leak de datos privados.
\end{itemize}