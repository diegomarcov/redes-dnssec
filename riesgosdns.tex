\chapter{Amenazas a DNS}

Existen muchos tipos de amenazas diferentes para el sistema DNS; muchas de ellas son instancias de problemas m\'as generales, pero tambi\'en existen algunas vulnerabilidades particulares al protocolo. Revisaremos a continuaci\'on las consideradas m\'as importantes\cite{rfc3833}.

\section{Intercepci\'on de paquetes}

Algunas de las amenazas m\'as sencillas para el protocolo son las diferentes formas de intercepci\'on de paquetes existentes. Si bien este problema est\'a lejos de ser exclusivo de DNS, el comportamiento particular del protocolo (en el que se env\'ian \textit{querys} o respuestas completas en un \'unico paquete UDP sin firma digital ni encriptaci\'on) lo hace muy facilmente vulnerable. 

Existen varios posibles ataques bajo la idea de intercepci\'on de paquetes. A continuaci\'on veremos algunos escenarios posibles\footnote{Los ataques presentados solo ser\'an vistos desde el punto de vista del servicio de resoluci\'on de nombres. Esto no implica que no posean otros \'angulos de ataque.}, y en el siguiente cap\'itulo analizaremos de qu\'e manera DNSSEC introduce confianza en los paquetes recibidos.

\subsection{MITM}

Este ataque explota el protocolo ARP. La idea b\'asica es que el atacante
env\'ia paquetes ARP armados de forma tal que identifique a s\'i mismo o a un
host controlado por el atacante como Puerta de Enlace Predeterminada de la red
local, lo que hace que todos los paquetes que est\'en dirigidos hacia alg\'un
host fuera de la LAN caigan en manos del atacante. De esta forma, las consultas
DNS pueden ser interceptadas y el atacante termina por controlar hacia d\'onde
se dirige la v\'ictima en todo momento. Este tipo de ataques son llamados
\textit{Man In The Middle}.

Una situaci\'on similar se presenta cuando una instituci\'on implementa un ``portal
cautivo'' para los usuarios de la red para acceso a internet. Esto no es un
ataque propiamente dicho, pero las consultas de los usuarios son interceptadas
y redireccionadas hacia alg\'un sitio de la instituci\'on en cuesti\'on.

\subsection{Intercepci\'on DHCP}

Los ataques MITM necesitan una cierta elaboraci\'on. Adem\'as de montar los
servidores a los cuales se redireccionar\'a a la v\'ictima, se tiene que contar
con software que se encargue de realizar el cambio de los paquetes necesarios en
tiempo real, lo cual no es una tarea trivial. 

Otra forma de realizar un ataque similar involucra el montado de un servidor
DHCP de forma que la v\'ictima, a la hora de realizar el pedido de una
direcci\'on de IP, tome la respuesta del servidor DHCP del atacante. Esta
respuesta estar\'ia conformada por los servidores de DNS primario y secundario,
los cuales ser\'ian controlados por el atacante.

\section{ID guessing y predicci\'on de queries}

Dado que un paquete DNS contiene un campo de ID en el encabezado que est\'a conformado por \'unicamente 16 bits, existen solo $2^{32}$ combinaciones de ID entre cliente y servidor. Peor a\'un, en la pr\'actica (debido a firewalls u otras restricciones) es com\'un que el ID y el puerto UDP del cliente pueden ser predichos a partir del tr\'afico previo, lo que reduce el espacio de b\'usqueda a menos de $2^{16}$ posibilidades: muy poco contra ataques por fuerza bruta. Si bien requiere cierta adivinaci\'on por parte de los atacantes, es muy efectivo cuando el cliente act\'ua de manera predecible por situaciones particulares, y puede permitir que el atacante introduzca falsas respuestas.\\
Una vez m\'as, observaremos c\'omo DNSSEC permite detectar este tipo de ataques.

\section{Envenenamiento de Cach\'e}

Todo los ataques anteriores est\'an principalmente enfocados hacia el cliente que realiza la consulta DNS. El envenamiento de cach\'e, en cambio, est\'a enfocado en atacar al servidor que responde a las \textit{queries}.\\
Cuando un cliente realiza una consulta a un servidor DNS, m\'as all\'a del tipo de resoluci\'on que utilice, en general se consulta m\'as de un servidor, formando una cadena de consultas en base a las distintas respuestas recibidas. Cuando las mismas consultas se realizan reiteradas veces, las respuestas suelen almacenarse en un cach\'e.\\
Dado que esta cach\'e est\'a compuesta de respuestas a consultas a otros servidores DNS, la veracidad de dichas respuestas es de suma importancia. Si un atacante logra utilizar alguna de las t\'ecnicas vistas anteriormente, solo que esta vez con un servidor DNS como v\'ictima, puede lograr
``envenenar'' la cach\'e con respuestas inv\'alidas cuidadosamente construidas. De esta forma, los usuarios que consulten a este servidor por
alguno de los dominios que no pertenezcan a la zona de \'este, el servidor primero consultar\'a su cach\'e para ver si ya se realiz\'o la consulta
anteriormente y en caso de que as\'i sea utilizar\'a la respuesta almacenada. Si el atacante ``envenena'' la respuesta correspondiente a un
dominio popular como \texttt{www.google.com}, es muy posible que pueda controlar todo lo que los usuarios del servidor DNS ``infectado'' hagan en la red.

\section{Traici\'on de un servidor de confianza}
Este tipo de amenaza es una variaci\'on de los ataques de intercepci\'on de paquetes, con la diferencia de que en este caso, el cliente env\'ia voluntariamente los pedidos al atacante. Dado que muchas m\'aquinas est\'an configuradas \'unicamente con \textit{stub resolvers}, y utilizan servidores de confianza para que hagan las \textit{queries} DNS en su lugar, si alguno de estos servidores no es tan confiable como aparenta, es posible que un atacante utilice maliciosamente dicho servidor enviando respuestas falsas a las \textit{queries} del cliente.\\
La defensa ante este tipo de ataques es la misma que para la intercepci\'on de paquetes: el cliente debe revisar por s\'i mismo las firmas digitales de DNSSEC, lo que a su vez puede resultar problem\'atico porque requiere conocer algunas claves p\'ublicas a las que el cliente no siempre tiene facil acceso.


\section{Denegaci\'on de servicio}
Como cualquier tipo de servicio en la red, DNS es vulnerable a los ataques de tipo Denial of Service (ataques que causan que un servicio sea inaccesible a los usuarios provocando la p\'erdida de la conectividad de la red o la sobrecarga de los recursos computacionales del sistema de la v\'ictima\cite{dos06}). Lamentablemente, DNSSEC no s\'olo no ayuda a resolver este problema, sino que puede empeorarlo considerablemente debido tanto a que la utilizaci\'on de firmas digitales requiere un mayor procesamiento por paquete, como a que la cantidad de paquetes enviados a la red es mayor.


\section{Determinaci\'on segura de nombres de dominio no existentes}

Uno de los grandes problemas que se presentan a la hora de plantear el modelo
de protecci\'on de DNSSEC es c\'omo corroborar de forma segura la no existencia de
un dominio en una consulta. Si ning\'un checkeo es realizado, resulta posible
interceptar una respuesta a un \textit{query} y descartar alg\'un RR de \'esta sin que
el cliente de la consulta pueda saberlo, simulando de esta manera la no existencia de un dominio arbitrario.\\
Esta situaci\'on hace que deba existir alguna forma segura de corroborar qu\'e dominios
efectivamente existen y cu\'ales no. Una posibilidad es mantener una base de
datos de dominios que no hay sido registrados, pero esto no representa una
soluci\'on viable.

\section{Wildcards}

Los \textit{wildcards} son utilizados para crear patrones similares a
expresiones regulares que permiten sintetizar el nombre de un dominio, y
evaluarlo din\'amicamente \cite{rfc1034}. El problema que esto trae est\'a
relacionado con el poder criptogr\'afico de autentificar un conjunto de datos
extra\'idos desde un servidor DNS: si un RR contiene wildcards, en la respuesta a
un query estos ser\'an ``instanciados'' y las firmas realizadas sobre esos datos
por las autoridades detr\'as del servidor no necesariamente seguir\'an siendo
v\'alidas. La integridad de los datos ha sido cambiada, y no es posible autentificar de forma
directa los datos con las t\'ecnicas criptogr\'aficas b\'asicas
establecidas para DNSSEC.
