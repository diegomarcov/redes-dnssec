\chapter{Amenazas a DNS}
Existen muchos tipos de amenazas diferentes para el sistema DNS; muchas de ellas son instancias de problemas más generales, pero también existen algunas vulnerabilidades particulares al protocolo. Revisaremos a continuación las consideradas más importantes\cite{rfc3833}.
\section{Intercepción de paquetes}
Algunas de las amenazas más sencillas para el protocolo son las diferentes formas de intercepción de paquetes existentes. Si bien este problema está lejos de ser exclusivo de DNS, el comportamiento particular del protocolo (en el que se envían \textit{querys} o respuestas completas en un único paquete UDP sin firma digital ni encriptación) lo hace muy facilmente vulnerable. En el siguiente capítulo analizaremos de qué manera DNSSEC introduce confianza en los paquetes recibidos.
\section{ID guessing y predicción de querys}
Dado que el campo de ID en el encabezado está conformado por únicamente 16 bits, existen solo 2**32 combinaciones de ID entre cliente y servidor. Peor aún, en la práctica (debido a firewalls u otras restricciones) es común que el ID y el puerto UDP del cliente pueden ser predichos a partir del tráfico previo, lo que reduce el espacio de búsqueda a menos de 2**16 posibilidades: muy poco contra ataques por fuerza bruta. Si bien requiere cierta adivinación por parte de los atacantes, es muy efectivo cuando el cliente actúa de manera predecible por situaciones particulares, y puede permitir que el atacante introduzca falsas respuestas.\\
Una vez más, observaremos cómo DNSSEC permite detectar este tipo de ataques.
\section{Name Chaining}
Es un tipo de amenaza particular al protocolo DNS, y son un subconjunto de los ataques llamados \" cache poisoning\" . .....
\section{Betrayal By Trusted Server}
\section{Denial of Service}
\section{Authenticated Denial of Domain Names}
\section{Wildcards}
